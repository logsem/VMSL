\documentclass{article}
\usepackage[utf8]{inputenc}
\usepackage[english]{babel}
\usepackage[a4paper,tmargin=0.75in,bmargin=1in,showframe=false]{geometry}
\usepackage{listings}
\usepackage{xcolor}
\usepackage{pgf-umlsd}
\usepackage{afterpage}
\usepackage{hyperref}
\usepackage{amsmath}
\usepackage{mathpartir}
\usepackage{hyperref}
\usepackage{todonotes}
\usepackage{graphicx}
\usepackage{kpfonts}
\usepackage{array}
\usepackage{changepage}
\usepackage[T1]{fontenc}
\usepackage{../iris}

\newenvironment*{returnanother}[6][1]{
  \stepcounter{seqlevel}
  \stepcounter{callevel} % push
  \path
  (#2)+(0,-\theseqlevel*\unitfactor-0.7*\unitfactor) node (cf\thecallevel) {}
  (#4.\threadbias)+(0,-\theseqlevel*\unitfactor-0.7*\unitfactor) node (ct\thecallevel) {};
  \draw[->,>=triangle 60] ({cf\thecallevel}) -- (ct\thecallevel)
  node[midway, above] {#3};
  \def\l\thecallevel{#1}
  \def\f\thecallevel{#6}
  \def\t\thecallevel{#4}
  \def\returnvalue{#5}
  \tikzstyle{threadstyle}+=[instcolor#2]
    \addtocounter{seqlevel}{1}
  \path
  (\f\thecallevel)+(0,-\theseqlevel*\unitfactor-0.7*\unitfactor) node (rf\thecallevel) {}
  (\t\thecallevel.\threadbias)+(0,-\theseqlevel*\unitfactor-0.7*\unitfactor) node (rt\thecallevel) {};
  \draw[dashed,->,>=angle 60] ({rt\thecallevel}) -- (rf\thecallevel)
  node[midway, above]{\returnvalue};
  \drawthread{ct\thecallevel}{rt\thecallevel}
}
{
\addtocounter{seqlevel}{\thecallevel}
%   \path
%   (\f\thecallevel.\threadbias)+(0,-\theseqlevel*\unitfactor+\unitfactor+\l\thecallevel*\unitfactor-0.7*\unitfactor) node (rrf\thecallevel) {}
%   (\f\thecallevel.\threadbias)+(0,-\theseqlevel*\unitfactor-0.7*\unitfactor) node (rff\thecallevel) {};
%  \drawthread{rrf\thecallevel}{rff\thecallevel}
  \addtocounter{callevel}{-1} % pop
  \addtocounter{seqlevel}{-1}
}



% notations
\newcommand*{\defined}{\triangleq\ }
\newcommand*{\maps}{\rightarrow}
\newcommand*{\derived}{\Coloneqq\ }
\newcommand*{\sem}[1]{\llbracket {#1} \rrbracket}
\newcommand*{\append}{\scalebox{0.8}{+\kern-0.4ex+}}


% definitions
\newcommand*{\CONF}{\text{ExecConf}}
\newcommand*{\STATE}{\text{State}}
\newcommand*{\MEM}{\text{GlobalMem}}
\newcommand*{\SSS}{\text{ShareStates}}
\newcommand*{\PID}{\text{PID}}
\newcommand*{\PT}{\text{PageTable}}
\newcommand*{\AS}{\text{AccessState}}
\newcommand*{\OS}{\text{OwnershipState}}
\newcommand*{\REGS}{\text{Registers}}
\newcommand*{\ADDR}{\text{Address}}
\newcommand*{\WORD}{\text{Word}}
\newcommand*{\VMID}{\text{VMID}}
\newcommand*{\REGNAMES}{\text{RegisterName}}
\newcommand*{\MODE}{\text{ExecMode}}
\newcommand*{\DONE}{\text{DoneState}}
\newcommand*{\INSTR}{\text{Instruction}}
\newcommand*{\MB}{\text{MailBox}}

% parameters
\newcommand*{\PABITS}{\text{ADDR\_BITS}}
\newcommand*{\PPBITS}{\text{PAGE\_BITS}}
\newcommand*{\PPIDBITS}{\text{PID\_BITS}}
\newcommand*{\PAMAX}{\text{ADDR\_MAX}}
\newcommand*{\PPMAX}{\text{PAGE\_MAX}}
\newcommand*{\PPIDMAX}{\text{PID\_MAX}}
\newcommand*{\PWBITS}{\text{WORD\_BITS}}
\newcommand*{\PWMAX}{\text{WORD\_MAX}}
\newcommand*{\PVMMAX}{\text{VM\_MAX}}

% FFA functions
\newcommand*{\INT}{\text{INTERRUPT}}
\newcommand*{\MSGS}{\text{MSG\_SEND}}
\newcommand*{\MSGW}{\text{MSG\_WAIT}}
\newcommand*{\YIELD}{\text{YIELD}}
\newcommand*{\BUSY}{\text{BUSY}}
\newcommand*{\SUCC}{\text{SUCCESS}}
% instructions
\newcommand*{\instrm}[1]{\mathtt{#1}}
\newcommand*{\instr}[1]{\texttt{#1}}


% expressions
\newcommand*{\EI}[1]{\mathtt{ExecInstr} \; {#1}}
\newcommand*{\RP}[1]{\mathtt{Repeat} \; {#1}}
\newcommand*{\DN}[1]{\mathtt{Done} \; {#1}}
\newcommand*{\NXT}[1]{\mathtt{Next} \; {#1}}

% functions
\newcommand*{\decode}{\text{decode}}
\newcommand*{\valida}{\text{valida}}
\newcommand*{\updpc}{\text{updpc}}
\newcommand*{\pid}{\text{pid}}
\newcommand*{\owned}{\text{owned}}
\newcommand*{\efid}[1]{\text{efid(\textsf{{#1}})}}


% SOME/NONE
\newcommand{\SOME}{\mathtt{Some}}
\newcommand{\NONE}{\mathtt{None}}

\lstset{
  backgroundcolor=\color{white},   % choose the background color; you must add \usepackage{color} or \usepackage{xcolor}; should come as last argument
  captionpos=t,                    % sets the caption-position to bottom
  commentstyle=\color{green},    % comment style
  keepspaces=true,                 % keeps spaces in text, useful for keeping indentation of code (possibly needs columns=flexible)
  keywordstyle=\color{blue},       % keyword style
  morekeywords={add,sub, mov, hvc, cmp, str, ldr, bne,halt, fail},            % if you want to add more keywords to the set
  rulecolor=\color{black},         % if not set, the frame-color may be changed on line-breaks within not-black text (e.g. comments (green here))
  title=\lstname                   % show the filename of files included with \lstinputlisting; also try caption instead of title
}


\begin{document}

\section{Scheduler $\rightarrow$ secondary VM}

Initially, VM 0 has both ownership and exclusive access to page $p$. After a complete lending transaction it loses an access to this page, and passes exclusive access to VM 1. After that, VM 1 relinquish access to the page, and the initial owner reclaims access.

More precisely, VM 0 initiates a memory lending transaction by populating a memory transaction descriptor (within its TX buffer) with receivers' IDs, PIDs, and some auxiliary information. R0 is populated with a function id of FFA\_MEM\_LEND, R1 is populated with a length of a memory descriptor. FFA\_MEM\_LEND returns a handle for this transaction. At this point, VM 0 owns the page but loses access to it. After the transaction initiation, VM 0 populates the memory transaction descriptor with the handle, and sends it via FFA\_MSG\_SEND (perhaps, without endpoint array), and transfers control to VM 1. VM 1 copies the memory transaction descriptor from its RX to TX buffer, and finishes the transaction by calling FFA\_MEM\_RETRIEVE. Now, VM 1 has exclusive access to the page. After that, VM 1 manipulate its newly acquired memory, relinquishes its access to it, and yields control back to VM 0. VM 0 reclaims access to the page.





\begin{figure}[hbt!]
\centering
\begin{sequencediagram}
    \newinst{Hyp}{Hypervisor}
    \newinst[3]{VM0}{VM 0}
    \newinst[3]{VM1}{VM 1}

    \begin{call}{VM0}{MEM\_LEND (1)}{Hyp}{SUCCESS}
    \end{call}

    \begin{call}{VM0}{MSG\_SEND (1)}{Hyp}{}
    \end{call}

    \begin{returnanother}{VM0}{RUN (1)}{Hyp}{}{VM1}
    \end{returnanother}

    \begin{call}{VM1}{MSG\_WAIT}{Hyp}{MSG\_SEND}
    \end{call}

    \begin{call}{VM1}{MEM\_RETRIEVE\_REQ}{Hyp}{MEM\_RETRIEVE\_RESP}
    \end{call}

    \begin{call}{VM1}{MEM\_RELINQUISH}{Hyp}{SUCCESS}
    \end{call}

    \begin{returnanother}{VM1}{YIELD}{Hyp}{}{VM0}
    \end{returnanother}

    \begin{call}{VM0}{MEM\_RECLAIM}{Hyp}{SUCCESS}
    \end{call}

\end{sequencediagram}

\end{figure}



\clearpage

\begin{lstlisting}[caption={VM 0}]
# save 42
mov R0, addr(p, 0)              # p address
mov R1, 42                      # payload
str R0, R1                      # write payload to p

# mem_lend
# arguments
mov R0, encode_fid(MEM_LEND)    # function id (FFA_MEM_LEND)
mov R1, 8                       # Total length of the
                                # memory transaction descriptor
# VMID of the Sender
mov R2, addr(tx, 0)             # TX buffer address
mov R3, 0
str R2, R3

# flag
mov R2, addr(tx, 1)             # TX buffer address (offset 1)
mov R3, 0                       # not zeroed
str R2, R3

# handle (mbz)
mov R2, addr(tx, 2)             # TX buffer address (offset 2)
mov R3, 0
str R2, R3

# tag (extra information)
mov R2, addr(tx, 3)             # TX buffer address (offset 3)
mov R3, 0
str R2, R3

# access descriptor count
mov R2, addr(tx, 4)             # TX buffer address (offset 4)
mov R3, 1
str R2, R3

# VMID of the first Receiver
mov R2, addr(tx, 5)             # TX buffer address (offset 5)
mov R3, 1
str R2, R3

# offset to the memory region descriptor
mov R2, addr(tx, 6)             # TX buffer address (offset 6)
mov R3, 2
str R2, R3

# number of pages
mov R2, addr(tx, 7)             # TX buffer address (offset 7)
mov R3, 1
str R2, R3

# base address of the donated page
mov R2, addr(tx, 8)             # TX buffer address (offset 8)
mov R3, addr(p, 0)              # p address
str R2, R3


hvc            # [0].pt[p] = (O, NA)
               # [1].pt[p] = (!O, NA)
               # sss[handler] =
               # (0, flag, tag, inl (1, p), MEM_LEND)

# R2 is populated with the transaction handler at this point
# save it to the discriptor
mov R3, addr(tx, 2)          # TX buffer address (offset 2)
str R3, R2

mov R4, R2                   # save the handler for later

# msg_send to 1
mov R0, encode_fid(MSG_SEND)    # function id (FFA_MSG_SEND)
mov R1, 1                       # VMID of the Receiver
mov R2, 8                       # message length
hvc

# run 1
mov R0, encode_fid(FFA_RUN)
mov R1, 1
hvc

# reclaim
mov R0, encode_id(MEM_RECLAIM)
mov R2, R4                       # handler
mov R3, 0                        # flag
hvc            # [0].pt[p] = (O, EA)
               # [1].pt[p] = (!O, NA)
               # find(sss, handler) = False
\end{lstlisting}

\begin{lstlisting}[caption={VM 1}]
# retrieve msg length
mov R0, encode_fid(MSG_WAIT)   # function id (FFA_MSG_WAIT)
hvc

# R1 is populated with length of the message at this point

# Copy RX to TX
mov R0, 0             # value to compare with
mov R2, loop          # jump point
mov R3, 1             # step
mov R4, addr(rx, 0)   # RX buffer base address
mov R5, addr(tx, 0)   # TX buffer base address
loop:
sub R1, R1, R3
cmp R1, R0
ldr R6, R4
str R5, R6
add R4, R4, R3
bne R2

mov R0, encode_fid(MEM_RETRIEVE_REQ) # function id (FFA_MEM_RETRIEVE_REQ)
mov R1, 4                            # Total length of the
                                     # memory transaction descriptor in bytes
# access descriptor count (must be 0 for mem_retrieve)
mov R2, addr(tx, 4)                  # TX buffer address (offset 4)
mov R3, 0
str R2, R3
hvc            # [0].pt[p] = (O, NA)
               # [1].pt[p] = (!O, EA)
               # sss[handler] =
               # (0, flag, tag, inl (1, p), MEM_LEND)

# save 24
mov R0, addr(p, 0)              # p address
mov R1, 24                      # payload
str R0, R1                      # write payload to p

# handler for relinquishing
mov R0, encode_fid(MEM_RELINQUISH)
mov R1, addr(tx, 2)
ldr R2, R1                      # save the old handler
mov R1, addr(tx, 1)
ldr R3, R1                      # save the old flag
mov R1, addr(tx, 0)             # TX buffer address
str R1, R2                      # handler for a new descriptor
mov R1, addr(tx, 1)
str R1, R3                      # flag for a new descriptor
hvc            # [0].pt[p] = (O, NA)
               # [1].pt[p] = (!O, NA)
               # sss[handler] =
               # (0, flag, tag, inl (1, p), MEM_LEND)

# yield
mov R0, encode_fid(YIELD)     # function id (FFA_YIELD)
hvc
\end{lstlisting}

\clearpage
According to the FF-A specification, parties participate in a transaction have responsibilities. For instance, the lender in a lending transaction must 1. support for handling all errors code; 2. send necessary information to borrowers if the FF-A call completes successfully; 3. treat the lent memory region as being inaccessible until a successful FFA\_MEM\_RECLAIM invocation. Thus, we tweaked the example above to filfull those requirements.
\begin{lstlisting}[caption={VM 0}]
# save 42
mov R0, addr(p, 0)              # p address
mov R1, 42                      # payload
str R0, R1                      # write payload to p

# mem_lend
# arguments
mov R0, encode_fid(MEM_LEND)    # function id (FFA_MEM_LEND)
mov R1, 8                       # Total length of the
                                # memory transaction descriptor
# VMID of the Sender
mov R2, addr(tx, 0)             # TX buffer address
mov R3, 0
str R2, R3

# flag
mov R2, addr(tx, 1)             # TX buffer address (offset 1)
mov R3, 0                       # not zeroed
str R2, R3

# handle (mbz)
mov R2, addr(tx, 2)             # TX buffer address (offset 2)
mov R3, 0
str R2, R3

# tag (extra information)
mov R2, addr(tx, 3)             # TX buffer address (offset 3)
mov R3, 0
str R2, R3

# access descriptor count
mov R2, addr(tx, 4)             # TX buffer address (offset 4)
mov R3, 1
str R2, R3

# VMID of the Receiver
mov R2, addr(tx, 5)             # TX buffer address (offset 5)
mov R3, 1
str R2, R3

# offset to the memory region descriptor
mov R2, addr(tx, 6)             # TX buffer address (offset 6)
mov R3, 2
str R2, R3

# number of pages
mov R2, addr(tx, 7)             # TX buffer address (offset 7)
mov R3, 1
str R2, R3

# base address of the donated page
mov R2, addr(tx, 8)             # TX buffer address (offset 8)
mov R3, addr(p, 0)              # p address
str R2, R3

hvc            # [0].pt[p] = (O, NA)
               # [1].pt[p] = (!O, NA)
               # sss[handler] =
               # (0, flag, tag, inl (1, p), MEM_LEND)


# check if the invocation was successful
# jump to err for error code handling
mov R8, encode_fid(SUCC)
cmp R0 R8
mov R8 err
bne R8

# R2 is populated with the transaction handler at this point
# save it to the discriptor
mov R3, addr(tx, 2)          # TX buffer address (offset 2)
str R3, R2

mov R9, R2                   # save the handler for later

send:
# msg_send to 1
mov R0, encode_fid(MSG_SEND)    # function id (FFA_MSG_SEND)
mov R1, 1                       # VMID of the Receiver
mov R2, 8                       # message length
hvc

# check if the invocation was successful
# try again if failed
mov R8, encode_fid(SUCC)
cmp R0 R8
mov R8 send
bne R8

run:
# run 1
mov R0, encode_fid(FFA_RUN)
mov R1, 1
hvc

# check if the invocation was successful
# halt if failed
mov R8, encode_fid(YIELD)
cmp R0 R8
mov R8 err
bne R8


# reclaim
mov R0, encode_id(MEM_RECLAIM)
mov R2, R9                       # handler
mov R3, 0                        # flag
hvc            # [0].pt[p] = (O, EA)
               # [1].pt[p] = (!O, NA)
               # find(sss, handler) = False

# check if the invocation was successful
# halt if failed
mov R8, encode_fid(SUCC)
cmp R0 R8
mov R8 fail
bne R8

halt

# fail if hypervisor returns an error
err:
fail

\end{lstlisting}

\begin{lstlisting}[caption={VM 1}]
pool:
# retrieve msg length
mov R0, encode_fid(MSG_POLL)   # function id (FFA_MSG_POLL)
hvc

# check if the invocation was successful
# halt if failed
mov R8, encode_fid(SUCC)
cmp R0 R8
mov R8 pool
bne R8

# R1 is populated with length of the message at this point

# Copy RX to TX
mov R0, 0             # value to compare with
mov R2, loop          # jump point
mov R3, 1             # step
mov R4, addr(rx, 0)   # RX buffer base address
mov R5, addr(tx, 0)   # TX buffer base address
loop:
sub R1, R1, R3
cmp R1, R0
ldr R6, R4
str R5, R6
add R4, R4, R3
bne R2

mov R0, encode_fid(MEM_RETRIEVE_REQ) # function id (FFA_MEM_RETRIEVE_REQ)
mov R1, 4                            # Total length of the
                                     # memory transaction descriptor in bytes
# access descriptor count (must be 0 for mem_retrieve)
mov R2, addr(tx, 4)                  # TX buffer address (offset 4)
mov R3, 0
str R2, R3
hvc            # [0].pt[p] = (O, NA)
               # [1].pt[p] = (!O, EA)
               # sss[handler] =
               # (0, flag, tag, inl (1, p), MEM_LEND)

# check if the invocation was successful
# halt if failed
mov R8, encode_fid(MEM_RETRIEVE_RESP)
cmp R0 R8
mov R8 err
bne R8

# save 24
mov R0, addr(tx, 8)              # p address
mov R1, 24                      # payload
str R0, R1                      # write payload to p

# handler for relinquishing
mov R0, encode_fid(MEM_RELINQUISH)
mov R1, addr(tx, 2)
ldr R9, R1                      # save the old handler
mov R1, addr(tx, 1)
ldr R10, R1                      # save the old flag
mov R1, addr(tx, 0)             # TX buffer address
str R1, R9                      # handler for a new descriptor
mov R1, addr(tx, 1)
str R1, R10                      # flag for a new descriptor
hvc            # [0].pt[p] = (O, NA)
               # [1].pt[p] = (!O, NA)
               # sss[handler] =
               # (0, flag, tag, inl (1, p), MEM_LEND)

# check if the invocation was successful
# halt if failed
mov R8, encode_fid(SUCC)
cmp R0 R8
mov R8 err
bne R8

# yield
mov R0, encode_fid(YIELD)     # function id (FFA_YIELD)
hvc

# halt if hypervisor returns an error
err:
fail
\end{lstlisting}


\newcommand*{\reg}[3]{\textsf{{#1}} \Rightarrow_{#2}{#3}}
\newcommand*{\adr}[2]{{#1} \hookrightarrow {#2}}
\newcommand*{\prog}[2]{{#1} \hookrightarrow \text{Prog}({#2})}
\newcommand*{\page}[2]{{#1} \hookrightarrow \text{Page}(#2)}
\newcommand*{\rxpage}[2]{\text{RX}_{#1}({#2})}
\newcommand*{\txpage}[2]{\text{TX}_{#1}({#2})}
\newcommand*{\sss}[1]{\text{ShareStates}({#1})}
\newcommand*{\own}[2]{\text{Owned}_{#1}(\{{#2}\})}
\newcommand*{\access}[2]{\text{Accessible}_{#1}(\{{#2}\})}
\newcommand*{\excl}[2]{\text{Exclusive}_{#1}(\{{#2}\})}
\newcommand*{\isEven}[1]{\text{isEven}({#1})}

\newcommand*{\oneinstr}[2]{&{#1}\\&\texttt{{#2}}\\}
\section{Specification}

\begin{align*}
 & \hoare{\begin{array}{l}
            \reg{PC}{0}{a_{i}^{0}} \asts \prog{a_{i}^{0}}{prog_{0}} \asts \\
             \reg{R1}{0}{-} \asts  \reg{R2}{0}{-}\asts  \reg{R3}{0}{-}\asts  \reg{R8}{0}{-}\asts  \reg{R9}{0}{-}\asts\\
            \own{0}{\pid(a_{i}^{0}), p}\asts \access{0}{\pid(a_{i}^{0}), p} \asts \excl{0}{\pid(a_{i}^{0}), p} \asts\\
           \txpage{0}{a_{tx}} \asts \adr{\pid(a_{tx})}{ws_{tx}} \asts \\
            \reg{PC}{1}{a_{i}^{1}} \asts \prog{a_{i}^{1}}{prog_{1}} \asts \\
            \reg{R0}{1}{-} \asts  \reg{R1}{1}{-} \asts  \reg{R2}{1}{-}\asts  \reg{R3}{1}{-}\asts  \reg{R8}{1}{-}\asts  \reg{R9}{1}{-}\asts\\
            \own{1}{\pid(a_{i}^{1})}\asts \access{1}{\pid(a_{i}^{1})} \asts \excl{1}{\pid(a_{i}^{1})} \asts\\
           \knowInv{\gamma_{1}}{\exists ws. \adr{p}{ws}\asts \isEven{ws}} \asts \knowInv{\gamma_{2}}{I(p,a_{rx})}
  \end{array}}
  {\textsf{Start~}[0,1]| -1}
  {v.v=\textsf{Done~Halt}} \\
 & \text{where}\\
  & I(p,a_{rx}) = \begin{array}{l}
          (\reg{R0}{0}{-} \asts \rxpage{1}{a_{rx},\FALSE, -,-} \asts \adr{\pid(a_{rx})}{-} \asts \sss{[]} \asts \ownGGhost{1}) \lor \\
                    \left(\begin{array}{ll}\exists w_{h}.& \reg{R0}{0}{\efid{RUN}} \asts \rxpage{1}{a_{rx},\TRUE,9,0} \asts \\
                    &\adr{\pid(a_{rx})}{[0,0,w_{h},0,1,1,2,1,p,0,...]} \asts \\ & \sss{[w_{h}\leftarrow (0,0,0,[],[p],\efid{MEM\_LEND})]} \asts \ownGGhost{0}\end{array}\right)\lor\\
           \left(\begin{array}{l}\reg{R0}{0}{\efid{YIELD}} \asts \rxpage{1}{a_{rx},\FALSE, -,-} \asts \adr{\pid(a_{rx})}{-} \asts \\ \sss{[w_{h}\leftarrow (0,0,0,[],[p],\efid{MEM\_LEND})]}\asts \ownGGhost{0}\end{array}\right)
          \end{array}
\end{align*}



\section{Proof Sketch}

\begin{align*}
  \oneinstr{
test
  }{mov R0, addr(p, 0)}
 \oneinstr{
  test
  }{mov R1, 42}
\oneinstr{
  test
  }{str R0, R1}
\oneinstr{
  test
  }{mov R0, encode\_fid(MEM\_LEND)}
\oneinstr{
  test
  }{mov R1, 8}
\oneinstr{
  test
  }{mov R2, addr(tx, 0)}
\oneinstr{
  test
  }{mov R3, 0}
\oneinstr{
  test
  }{str R2, R3}
\oneinstr{
  test
  }{mov R2, addr(tx, 1)}
\oneinstr{
  test
  }{mov R3, 0}
\oneinstr{
  test
  }{str R2, R3}
  \end{align*}


\end{document}
